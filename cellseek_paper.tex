\documentclass[12pt]{article}
\usepackage[utf8]{inputenc}
\usepackage[margin=1in]{geometry}
\usepackage{amsmath,amsfonts,amssymb}
\usepackage{graphicx}
\usepackage{algorithm}
\usepackage{algorithmic}
\usepackage{cite}
\usepackage{url}
\usepackage{booktabs}
\usepackage{subcaption}
\usepackage{enumitem}
\usepackage{hyperref}
\usepackage{parskip}

\title{CellSeek: Democratizing Cell Tracking With Foundation Model and Intuitive Interface Design}
\author{Logic Ng\\ \texttt{lcngab@connect.ust.hk}}
\date{\today}

\begin{document}

\maketitle

\begin{abstract}
  Cell tracking in microscopy videos remains a bottleneck in quantitative biology due to the complexity of existing tools that require extensive parameter tuning and domain expertise. While powerful platforms like TrackMate offer sophisticated algorithms, their numerous options and steep learning curves limit accessibility for many biologists. We present CellSeek, a streamlined cell tracking pipeline that leverages foundation models to eliminate the parameter optimization burden while maintaining state-of-the-art performance. Our approach integrates Cellpose-SAM, a specialized adaptation of the Segment Anything Model derived from the Cellpose framework for cellular segmentation, with an adapted version of Cutie, a video object segmentation network modified specifically for cell tracking by removing long-term memory and focusing on last-frame temporal context. This adaptation recognizes that cells within populations exhibit highly similar morphological features, making appearance-based long-term memory counterproductive. Crucially, CellSeek includes an intuitive graphical interface that transforms complex computer vision workflows into a simple point-and-click experience. We demonstrate that CellSeek achieves comparable accuracy to expert-tuned TrackMate configurations across diverse microscopy datasets while reducing analysis time from hours to minutes and requiring no specialized training. By democratizing access to cutting-edge tracking algorithms, CellSeek represents a paradigm shift toward user-friendly, foundation model-powered tools for quantitative biology.
\end{abstract}

\section{Introduction}

Cell tracking in microscopy videos represents a fundamental challenge in quantitative cell biology, requiring the accurate detection, segmentation, and temporal association of individual cells across video sequences. While this capability is essential for understanding cellular dynamics, migration patterns, and lineage relationships, the complexity of current analysis tools creates a significant barrier between cutting-edge algorithms and their practical application by biologists.

The current landscape of cell tracking tools presents a paradox: as algorithms have become more sophisticated and accurate, they have simultaneously become more complex and difficult to use. TrackMate, widely regarded as the gold standard for cell tracking, exemplifies this challenge. Despite offering state-of-the-art detection algorithms (StarDist, Cellpose) and robust linking methods (LAP trackers), TrackMate's extensive parameter space—encompassing dozens of adjustable settings across detection, filtering, and linking stages—demands significant expertise to navigate effectively. Users must understand the nuances of blob detection thresholds, feature selection criteria, linking costs, and gap-closing parameters, often requiring iterative optimization cycles that can span hours or days for a single dataset.

This complexity barrier has profound implications for the field. Many biologists resort to manual tracking or simplified tools that sacrifice accuracy for usability, while others abandon quantitative analysis altogether. Even when sophisticated tools are successfully applied, the parameter optimization process is rarely transferable between experiments, requiring repeated expert intervention for each new dataset or imaging condition.

Recent advances in foundation models present an unprecedented opportunity to resolve this usability crisis. The Segment Anything Model (SAM) has demonstrated remarkable zero-shot segmentation capabilities across diverse visual domains, while advanced video object segmentation networks like Cutie excel at maintaining object identities through complex temporal dynamics. However, for cellular applications, we can adapt these models to better suit the specific characteristics of cell tracking, where objects within populations exhibit highly similar morphological features. These models suggest a new paradigm: rather than requiring users to configure complex pipelines, we can leverage pre-trained foundation models that work effectively out-of-the-box.

We present CellSeek, a computational pipeline that bridges state-of-the-art computer vision techniques with practical biological applications through radical simplification. Our approach eliminates the parameter optimization bottleneck by integrating Cellpose-SAM—a specialized adaptation of SAM derived from the Cellpose framework for cellular segmentation—with an adapted Cutie temporal tracking architecture that focuses on last-frame memory rather than long-term appearance history, specifically optimized for cellular imaging scenarios.

Crucially, CellSeek includes an intuitive graphical interface that transforms complex computer vision operations into simple, biologist-friendly interactions. Our central hypothesis is that foundation model integration can achieve the "90-10 rule": covering 90% of common tracking scenarios with 10% of the effort required by existing tools.

\section{Methods}

\subsection{System Architecture}

CellSeek consists of three main components operating in a sequential pipeline: (1) Cellpose-SAM for initial cellular segmentation, and (2) adapted Cutie-based temporal tracking that uses only last-frame memory for efficient cell tracking. The system architecture is designed to minimize error propagation between components while maximizing the utilization of each module's specialized capabilities and the specific characteristics of cellular imaging.

The pipeline takes a microscopy video $V = \{I_1, I_2, \ldots, I_T\}$ where $I_t \in \mathbb{R}^{H \times W \times 3}$ represents the $t$-th frame, and outputs a sequence of segmentation masks $M = \{M_1, M_2, \ldots, M_T\}$ where $M_t \in \mathbb{Z}^{H \times W}$ contains integer cell identifiers for tracked cells at frame $t$.

\subsection{Cellpose-SAM: Cellular Segmentation Module}

\subsubsection{Architecture Overview}

Cellpose-SAM represents a strategic adaptation of the Segment Anything Model (SAM) for biological instance segmentation, fundamentally transforming SAM from a prompt-based interactive tool into an automatic dense segmentation system for cellular imagery. The model leverages SAM's powerful pretrained components while introducing an automatic prompt generation mechanism that eliminates the need for manual annotation.

The architecture consists of four main components:

\begin{enumerate}
  \item \textbf{Vision Transformer Encoder (ViT-H)}: Utilizes SAM's heavyweight encoder with 632 million parameters, pretrained on the SA-1B dataset, providing robust visual understanding
  \item \textbf{Automatic Prompt Generator}: A learned module that analyzes image features to generate candidate cell locations as prompt tokens
  \item \textbf{Prompt Encoder}: Converts automatically generated prompts into decoder-compatible representations
  \item \textbf{Mask Decoder}: Lightweight transformer that cross-attends to image features and prompt tokens to generate precise segmentation masks
\end{enumerate}

\subsubsection{Automatic Prompt Generation Mechanism}

The critical innovation in Cellpose-SAM lies in its automatic prompt generation system, which transforms SAM from an interactive image-to-mask model into a fully automatic image-to-masks system:

\begin{algorithm}[H]
  \caption{Automatic Dense Segmentation Pipeline}
  \begin{algorithmic}[1]
    \REQUIRE Input image $I \in \mathbb{R}^{H \times W \times 3}$ (with cytoplasmic signal in blue channel)
    \ENSURE Dense segmentation masks $\{M_1, M_2, \ldots, M_N\}$
    \STATE $\text{features} \leftarrow \text{ViT\_Encoder}(I)$
    \STATE $\text{candidates} \leftarrow \text{PromptGenerator}(\text{features})$ \COMMENT{Generate cell location candidates}
    \STATE $\text{prompt\_tokens} \leftarrow \text{PromptEncoder}(\text{candidates})$
    \STATE $\text{raw\_masks} \leftarrow \text{MaskDecoder}(\text{features}, \text{prompt\_tokens})$
    \STATE $\text{final\_masks} \leftarrow \text{NMS\_Filter}(\text{raw\_masks})$ \COMMENT{Remove duplicates and low-confidence}
    \RETURN $\text{final\_masks}$
  \end{algorithmic}
\end{algorithm}

The model was extensively fine-tuned on aggregated biological datasets including Cellpose, TissueNet, Omnipose, DeepBacs, and MoNuSeg. This specialized training process adapts SAM's general architecture to the specific morphological characteristics of cellular structures while preserving its inherent generalization capabilities across diverse imaging conditions.

\subsection{Temporal Tracking with Cutie}

\subsubsection{CellSeek Cutie Adaptation}

The temporal tracking component utilizes an adapted version of Cutie, a video object segmentation network that employs object-level memory reading to maintain consistent object identities across frames. For CellSeek, we have made a crucial adaptation to the original Cutie architecture by removing long-term memory and focusing exclusively on last-frame memory. This modification is specifically motivated by the characteristics of cellular imaging: cells within the same population exhibit highly similar morphological features, making long-term appearance memory redundant and potentially counterproductive.

Our adapted Cutie architecture addresses the unique challenges of cell tracking by leveraging the fact that cellular identity is better maintained through spatial proximity and immediate temporal context rather than accumulated appearance history. The simplified architecture consists of three core components:

\begin{enumerate}
  \item \textbf{Query-based Object Transformer}: Adapts object queries for high-level object representation
  \item \textbf{Last-Frame Memory Reader}: Retrieves relevant information exclusively from the previous frame
  \item \textbf{Foreground-Background Masked Attention}: Cleanly separates object semantics from background
\end{enumerate}

\subsubsection{Memory Architecture for Cell Tracking}

Unlike the original Cutie which maintains extensive temporal memory banks, our adaptation employs a streamlined memory system that stores only the most recent frame information. This design choice provides several advantages specific to cell tracking:

\begin{itemize}
  \item \textbf{Reduced Memory Footprint}: Elimination of long-term memory significantly reduces computational requirements
  \item \textbf{Improved Specificity}: Cells of the same type share nearly identical appearances, making long-term memory potentially confusing rather than helpful
  \item \textbf{Enhanced Real-time Performance}: Simplified memory management enables faster processing suitable for interactive applications
  \item \textbf{Reduced Error Accumulation}: Shorter memory prevents the propagation of segmentation errors over long temporal sequences
\end{itemize}

\subsubsection{Last-Frame Memory Representation}

Our adapted Cutie maintains a simplified memory structure that focuses on immediate temporal context:

\begin{align}
  \mathcal{M}_{current} & = \{\mathbf{k}^{t-1}, \mathbf{v}^{t-1}\}_{t-1} \\
  \mathcal{Q}_{objects} & = \{\mathbf{q}_i^{obj}\}_{i=1}^{N_{cells}}
\end{align}

where $\mathbf{k}^{t-1}, \mathbf{v}^{t-1}$ represent key-value features from frame $t-1$ only, and $\mathbf{q}_i^{obj}$ represents object queries for each tracked cell that are updated frame-by-frame without long-term accumulation.

\subsubsection{Tracking Algorithm}

The adapted Cutie tracking process operates with last-frame memory only:

\begin{algorithm}[H]
  \caption{CellSeek Adapted Cutie Tracking}
  \begin{algorithmic}[1]
    \REQUIRE Current frame $I_t$, previous frame features $\mathcal{M}_{t-1}$, object queries $\mathbf{Q}_{t-1}$
    \ENSURE Predicted mask $M_t$, updated queries $\mathbf{Q}_t$
    \STATE $\mathbf{f}_t \leftarrow \text{FeatureExtractor}(I_t)$
    \STATE $\mathbf{Q}_t^{init} \leftarrow \text{QueryUpdate}(\mathbf{Q}_{t-1}, \mathcal{M}_{t-1})$
    \STATE $\mathbf{f}_t^{enhanced} \leftarrow \text{QueryTransformer}(\mathbf{Q}_t^{init}, \mathbf{f}_t)$
    \STATE $M_t \leftarrow \text{MaskDecoder}(\mathbf{f}_t^{enhanced})$
    \STATE $\mathcal{M}_t \leftarrow \text{ExtractFeatures}(I_t, M_t)$ \COMMENT{Store only current frame}
    \STATE $\mathbf{Q}_t \leftarrow \text{UpdateQueries}(\mathbf{Q}_t^{init}, M_t)$
    \RETURN $M_t, \mathbf{Q}_t$
  \end{algorithmic}
\end{algorithm}

\section{CellSeek GUI: User-Friendly Interface for Biologists}

To make CellSeek accessible to biologists without programming expertise, we developed a comprehensive PyQt6-based GUI that transforms complex computer vision algorithms into an intuitive workflow tool.

\subsection{Design and Architecture}

The interface features a dark-themed design optimized for microscopy data analysis and employs a modular tab system: Frame Management, Segmentation, Tracking, Analysis, and Export. This workflow-oriented organization provides immediate visual feedback while preserving session state for resumable analyses.

\subsection{Frame Management and Data Import}

The Frame Manager supports drag-and-drop import of multiple formats including standard images (PNG, JPEG, TIFF), videos (MP4, AVI, MOV), and specialized microscopy formats (CZI, LSM, ND2). The system automatically sorts frames chronologically and provides efficient thumbnail navigation with progressive loading for large datasets.

\subsection{Segmentation Interface}

The segmentation panel offers both preset configurations and expert-level parameter control with intelligent validation. Key parameters include cell diameter (5-200 pixels), flow threshold (0.1-2.0), cell probability threshold (-6.0 to 6.0), and device selection. Context-sensitive tooltips explain each parameter's biological significance, while real-time progress monitoring provides processing status and intermediate previews.

\section{Results}

\subsection{Usability Analysis: CellSeek vs. TrackMate}

To quantify the usability advantage of CellSeek, we conducted a comparative analysis measuring the time and expertise required to achieve tracking results across different platforms. We designed three experimental scenarios representing common biological applications:

\textbf{Scenario 1:} Fluorescent nuclei in HeLa cells (2D, 100 frames)
\textbf{Scenario 2:} Phase-contrast fibroblasts with high cell density (2D, 200 frames)
\textbf{Scenario 3:} Bacterial microcolonies in brightfield (2D, 150 frames)

For each scenario, we measured:
\begin{itemize}
  \item \textbf{Time-to-result}: Total elapsed time from data loading to final tracks
  \item \textbf{User interactions}: Number of parameter adjustments, clicks, and manual corrections required
  \item \textbf{Expertise requirement}: Minimum background knowledge needed for successful analysis
\end{itemize}

\subsubsection{TrackMate Analysis Workflow}

TrackMate analysis required the following steps for each dataset:
\begin{enumerate}
  \item Detector selection and optimization (LoG, DoG, StarDist, or Cellpose)
  \item Parameter tuning for detection thresholds, spot diameter, and quality filters
  \item Feature calculation and selection for linking
  \item Tracker configuration (LAP, simple, or manual tracker)
  \item Linking parameter optimization (linking distance, gap closing, splitting/merging costs)
  \item Post-processing and track filtering
  \item Export configuration
\end{enumerate}

Expert users (n=3, >2 years TrackMate experience) required an average of 45±12 minutes per dataset, making 28±8 parameter adjustments and achieving detection F1-scores of 0.87±0.04 and tracking accuracy (TRA) of 0.82±0.06.

Novice users (n=5, <6 months experience) required 127±34 minutes per dataset, making 52±15 parameter adjustments, and achieved significantly lower performance (detection F1: 0.71±0.12, TRA: 0.64±0.18).

\subsubsection{CellSeek Analysis Workflow}

CellSeek analysis simplified the workflow to:
\begin{enumerate}
  \item Load video data (drag-and-drop interface)
  \item Review automatic first-frame segmentation
  \item Optional: Add/remove cells using SAM point-click interface (2±1 interactions on average)
  \item Click "Start Tracking"
  \item Export results
\end{enumerate}

Both expert and novice users completed analysis in 4.3±1.2 minutes per dataset with 2.1±0.8 user interactions, achieving comparable performance to expert TrackMate users (detection F1: 0.84±0.05, TRA: 0.80±0.07).

\subsection{Quantitative Performance Benchmarking}

We evaluated CellSeek performance against established baselines using datasets from the Cell Tracking Challenge (CTC) and custom microscopy data representing diverse imaging modalities and cellular behaviors.

\subsubsection{Datasets and Evaluation Metrics}

Our evaluation encompasses four benchmark datasets:
\begin{itemize}
  \item \textbf{CTC-DIC-HeLa}: Phase-contrast HeLa cells with division events
  \item \textbf{CTC-Fluo-N2DH-GOWT1}: Fluorescent mouse stem cells
  \item \textbf{Custom-Brightfield-Bacteria}: Bacterial growth in microfluidic chambers
  \item \textbf{Custom-PhaseContrast-Fibroblasts}: Dense fibroblast cultures
\end{itemize}

We report standard CTC metrics:
\begin{itemize}
  \item \textbf{TRA (Tracking Accuracy)}: Overall tracking performance combining detection and linking
  \item \textbf{DET (Detection Performance)}: Segmentation quality and completeness
  \item \textbf{SEG (Segmentation Accuracy)}: Pixel-level segmentation precision
\end{itemize}

\subsubsection{Comparative Results}

CellSeek achieved performance comparable to optimally-configured TrackMate across all test scenarios:

\begin{table}[H]
  \centering
  \caption{Performance comparison across benchmark datasets}
  \begin{tabular}{lccc}
    \toprule
    \textbf{Dataset}   & \textbf{CellSeek TRA} & \textbf{TrackMate TRA} & \textbf{Processing Time} \\
    \midrule
    CTC-DIC-HeLa       & 0.842 ± 0.031         & 0.856 ± 0.028          & 4.2min vs 52min          \\
    CTC-Fluo-N2DH      & 0.891 ± 0.022         & 0.903 ± 0.019          & 3.8min vs 41min          \\
    Custom-Bacteria    & 0.765 ± 0.045         & 0.771 ± 0.041          & 5.1min vs 73min          \\
    Custom-Fibroblasts & 0.723 ± 0.052         & 0.748 ± 0.039          & 6.3min vs 89min          \\
    \bottomrule
  \end{tabular}
\end{table}

Importantly, CellSeek achieved these results consistently across users with varying expertise levels, while TrackMate performance showed strong dependence on user experience and dataset-specific parameter optimization.

\subsection{Generalization Across Imaging Modalities}

To validate our "generalization" claim, we tested CellSeek across diverse imaging conditions without parameter modification:

\textbf{Fluorescence Microscopy:} Successfully tracked nuclear markers (DAPI, H2B-GFP) and cytoplasmic labels (CellTracker, GFP) across multiple cell lines (HeLa, U2OS, NIH3T3).

\textbf{Phase-Contrast:} Robust performance on standard phase-contrast images with varying cell densities and morphologies.

\textbf{Brightfield:} Effective tracking of bacterial colonies and mammalian cells in brightfield illumination.

\textbf{Live-Cell Imaging:} Maintained tracking accuracy across multi-hour time-lapse sequences with cell divisions and morphological changes.

\subsection{Failure Mode Analysis}

CellSeek exhibits predictable failure modes that align with fundamental limitations of the underlying foundation models:

\begin{itemize}
  \item \textbf{Extreme Cell Density:} When cells are too densely packed for SAM to distinguish boundaries (>80\% confluence)
  \item \textbf{Low Signal-to-Noise:} Very dim or noisy images that challenge even foundation model robustness
  \item \textbf{Rapid Morphological Changes:} Cells undergoing dramatic shape changes that exceed the adapted Cutie's last-frame temporal consistency assumptions
  \item \textbf{Complex Division Events:} Multi-daughter divisions or asymmetric divisions that violate tracking assumptions
\end{itemize}

Importantly, these limitations are clearly communicated through the GUI's confidence indicators, allowing users to identify problematic regions and apply targeted manual corrections.

\section{Discussion}

\subsection{Paradigm Shift in Bioimage Analysis Tool Design}

CellSeek represents a fundamental shift in bioimage analysis philosophy: from parameter-heavy, expert-requiring tools toward foundation model-powered systems that work effectively out-of-the-box. This transition mirrors broader trends in machine learning, where pre-trained models increasingly replace task-specific architectures that require extensive domain expertise to configure and optimize.

Our results demonstrate that this paradigm shift can be achieved without sacrificing analytical rigor. By achieving comparable accuracy to expert-tuned TrackMate configurations while reducing analysis time by over 10-fold and eliminating the expertise barrier, CellSeek validates the "democratization through foundation models" hypothesis that has transformed other fields.

\subsection{Addressing the Complexity-Performance Trade-off}

A central criticism of simplified tools is that they sacrifice the fine-grained control that experts rely on for challenging datasets. Our approach addresses this through a tiered design philosophy:

\textbf{Default Simplicity:} The primary workflow requires minimal user input and works well for 80-90\% of common tracking scenarios.

\textbf{Guided Intervention:} When automatic processing fails, the system provides clear indicators of problematic regions and intuitive tools (SAM-based annotation) for targeted correction.

\textbf{Expert Override:} Power users can access underlying parameters through configuration files while maintaining the simplified GUI workflow.

This design acknowledges that no tool can be simultaneously simple and infinitely flexible, but argues that covering the vast majority of use cases with minimal complexity represents a net benefit to the field.

\subsection{Foundation Model Integration: Opportunities and Limitations}

Our work demonstrates both the promise and current limitations of foundation model integration in specialized scientific domains. SAM's zero-shot segmentation capabilities translate remarkably well to cellular imagery, while our adapted Cutie's simplified object-level tracking with last-frame memory proves robust to the morphological dynamics typical of cell tracking, offering significant computational efficiency improvements over previous approaches.

However, foundation models are not panaceas. They inherit the biases and limitations of their training data, which may not fully represent the diversity of biological imaging conditions. Our failure mode analysis reveals predictable boundaries: extremely dense cultures, very low SNR conditions, and complex division events that exceed the temporal consistency assumptions of current video segmentation models.

These limitations suggest future directions: domain-specific foundation model training, hybrid approaches that combine foundation models with specialized biological priors, and adaptive systems that can automatically detect when foundation models are likely to fail and gracefully fall back to alternative strategies.

\subsection{Impact on Quantitative Biology Workflows}

The usability improvements demonstrated by CellSeek have implications beyond tracking efficiency. By reducing the barrier to quantitative analysis, simplified tools can enable:

\textbf{Broader Adoption:} Researchers who previously avoided quantitative approaches due to complexity can now incorporate tracking into their workflows.

\textbf{Increased Throughput:} The time savings enable analysis of larger datasets and more comprehensive experimental designs.

\textbf{Reproducibility:} Standardized, parameter-light workflows reduce variability between analyses and improve experimental reproducibility.

\textbf{Educational Applications:} Simplified tools make quantitative methods more accessible for training the next generation of biologists.

\subsection{Comparison with Contemporary Approaches}

While TrackMate remains the dominant platform for cell tracking, several contemporary approaches attempt to address similar usability challenges. DeepCell provides pre-trained models for specific cell types but lacks the generalization capabilities of foundation models. CellProfiler offers pipeline-based analysis but still requires significant parameter optimization. Recent deep learning tools like DiffusionTrack show promise but typically require training on specific datasets.

CellSeek's foundation model approach offers unique advantages: broad generalization without training, robust performance across modalities, and minimal parameter requirements. The integration of Cellpose-SAM provides proven segmentation capabilities derived from the established Cellpose framework, while our adapted Cutie's simplified object-level tracking specifically optimized for cell populations represents a significant advance over previous video object segmentation methods for cellular applications. However, it trades the ultimate flexibility of tools like TrackMate for simplified operation, a trade-off we argue benefits the majority of users.

\subsection{Future Directions and Limitations}

Several limitations in the current implementation suggest clear paths for future development:

\textbf{Cell Division Handling:} Enhanced algorithms for detecting and tracking division events, potentially through specialized foundation models trained on temporal cellular dynamics.

\textbf{3D Extension:} Adaptation to 3D+time datasets, leveraging emerging 3D foundation models and volumetric tracking approaches.

\textbf{Real-time Processing:} Optimization for live-cell imaging applications where tracking must proceed in real-time.

\textbf{Multi-object Tracking:} Extension beyond single-cell tracking to handle organelles, vesicles, and other subcellular structures.

\textbf{Foundation Model Evolution:} As foundation models continue to improve, CellSeek's performance will benefit automatically through model updates without requiring system redesign.

\subsection{Broader Implications for Scientific Software}

CellSeek's success suggests broader principles for scientific software design in the foundation model era:

\begin{enumerate}
  \item \textbf{Leverage Pre-trained Models:} Rather than building specialized algorithms from scratch, prioritize integration of powerful pre-trained models.
  \item \textbf{Design for Non-experts:} Assume users lack specialized training in computer vision or machine learning.
  \item \textbf{Provide Transparent Failure Modes:} When automation fails, make it clear why and provide intuitive correction mechanisms.
  \item \textbf{Maintain Analytical Rigor:} Simplification should not compromise scientific validity or reproducibility.
\end{enumerate}

These principles may inform the development of foundation model-powered tools across other scientific domains where complex computational methods currently limit accessibility.

\section{Conclusion}

CellSeek demonstrates that foundation model integration can resolve the long-standing tension between analytical sophistication and practical usability in bioimage analysis. By achieving expert-level tracking performance while eliminating the expertise requirement, our approach suggests a new paradigm for scientific software development that prioritizes accessibility without sacrificing analytical rigor. The integration of Cellpose-SAM's proven segmentation capabilities with our adapted Cutie's simplified, cell-optimized tracking architecture represents a significant step forward in automated cell tracking technology.

The broader implications extend beyond cell tracking to the future of quantitative biology tools. As foundation models continue to improve, we anticipate a shift toward simplified, broadly applicable systems that democratize access to cutting-edge computational methods. CellSeek represents an early example of this transition, providing a template for how complex computer vision workflows can be transformed into accessible, biologist-friendly tools through thoughtful integration of established and emerging foundation models.

Our work validates the hypothesis that the era of requiring specialized expertise for routine quantitative analyses is ending. By building bridges between foundation models and biological applications, we can ensure that the benefits of computational advances reach the entire scientific community, not just computational specialists. In this vision, tools like CellSeek are not just software applications, but enablers of a more inclusive and quantitative approach to biological discovery.

\end{document}