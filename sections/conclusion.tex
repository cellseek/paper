\documentclass[../cellseek_paper.tex]{subfiles}

\begin{document}

\section{Conclusion}

CellSeek demonstrates that foundation model integration can resolve the long-standing tension between analytical sophistication and practical usability in bioimage analysis. By achieving expert-level tracking performance while eliminating the expertise requirement, our approach suggests a new paradigm for scientific software development that prioritizes accessibility without sacrificing analytical rigor. The integration of Cellpose-SAM's proven segmentation capabilities with our adapted Cutie's simplified, cell-optimized tracking architecture represents a significant step forward in automated cell tracking technology.

The broader implications extend beyond cell tracking to the future of quantitative biology tools. As foundation models continue to improve, we anticipate a shift toward simplified, broadly applicable systems that democratize access to cutting-edge computational methods. CellSeek represents an early example of this transition, providing a template for how complex computer vision workflows can be transformed into accessible, biologist-friendly tools through thoughtful integration of established and emerging foundation models.

Our work validates the hypothesis that the era of requiring specialized expertise for routine quantitative analyses is ending. By building bridges between foundation models and biological applications, we can ensure that the benefits of computational advances reach the entire scientific community, not just computational specialists. In this vision, tools like CellSeek are not just software applications, but enablers of a more inclusive and quantitative approach to biological discovery.

\end{document}
